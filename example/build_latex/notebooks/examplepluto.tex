\newpage

\chapter{Chapter}

\section{Section}

\subsection{Subsection}

\subsubsection{Subsubsection}
My text here with \textit{italics}, with \textbf{bold}, and a  \href{https://davibarreira.github.io/}{link}.Adding some math expression here with $x=10$ and $y = 10^2 + 2*2$.
\begin{displaymath}
	d(\omega(t_0),\omega(t_1)) \leq \int^{t_1}_{t_0}g(s) ds.
\end{displaymath}
Adding some code like \lstinline[style=julia]{plots}. Note that the \lstinline[style=julia]{using plots}
\begin{lstlisting}[language=JuliaLocal, style=julia]
using PlutoUI
\end{lstlisting}

\begin{lstlisting}[language=JuliaLocal, style=julia]
begin
	using Plots
	y(x) = sin(x)
	plot(y,
		color=:blue)
end
\end{lstlisting}

\begin{figure}[H]
	\centering
	\includegraphics[width=0.8\textwidth]{./figures/examplepluto_figure1.png}
	\label{fig:examplepluto_figure1.png}

\end{figure}

\begin{lstlisting}[language=JuliaLocal, style=julia]
A = [10,10,10]
\end{lstlisting}

\begin{verbatim}
3-element Vector{Int64}:
 10
 10
 10
\end{verbatim}

\begin{lstlisting}[language=JuliaLocal, style=julia]
x = rand(10);
\end{lstlisting}

\begin{lstlisting}[language=JuliaLocal, style=julia]
x .+ 1
\end{lstlisting}

\begin{verbatim}
10-element Vector{Float64}:
 1.8970234587184498
 1.205897447894831
 1.271917710246342
 1.151002269359065
 1.1531993979058628
 1.6966757257018419
 1.046939839898896
 1.055353446889297
 1.6327116712841039
 1.117067446721321
\end{verbatim}

\begin{lstlisting}[language=JuliaLocal, style=julia]
set_theme!(theme_ggplot2())
\end{lstlisting}

\begin{lstlisting}[language=JuliaLocal, style=julia]
Makie.plot(x)
\end{lstlisting}

\begin{figure}[H]
	\centering
	\includegraphics[width=0.8\textwidth]{./figures/examplepluto_figure2.pdf}
	\label{fig:examplepluto_figure2.pdf}

\end{figure}

\begin{lstlisting}[language=JuliaLocal, style=julia]
PlutoUI.LocalResource(figurepath)
\end{lstlisting}

\begin{figure}[H]
	\centering
	\includegraphics[width=0.8\textwidth]{./figures/plotexample.png}
	\label{fig:/home/davibarreira/MEGA/EMAp/NotebookToLatex.jl/example/plotexample.png}

\end{figure}
